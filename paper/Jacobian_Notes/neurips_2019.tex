\documentclass{article}

%%%%%%%%
\usepackage{fullpage}
\usepackage[top=2cm, bottom=4.5cm, left=2.5cm, right=2.5cm]{geometry}
\usepackage{amsmath,amsthm,amsfonts,amssymb,amscd}
\usepackage{lastpage}
\usepackage{enumerate}
\usepackage{fancyhdr}
\usepackage{mathrsfs}
\usepackage{xcolor}
\usepackage{graphicx}
\usepackage{listings}
\usepackage{hyperref}
\usepackage{algorithm}
\usepackage{algorithmic}
\usepackage{braket}
\usepackage{caption}
\usepackage{subcaption}
\usepackage[ruled,vlined]{algorithm2e}
\newtheorem{theorem}{Theorem}
\usepackage{physics}



\newcommand{\floor}[1]{\left\lfloor #1 \right\rfloor}
\newcommand{\ceil}[1]{\left\lceil #1 \right\rceil}

\newcommand{\norm}[1]{\left\lVert#1\right\rVert}

\newcommand{\mtx}[1]{\mathbf{#1}}
\newcommand{\m}[1]{\mathbf{#1}}
\newcommand{\norm}[1]{\left\lVert#1\right\rVert}
\newcommand{\E}{\mathbb{E}}
\newcommand{\delE}{\nabla_{\theta}\mathbb{E}_{p(x;\theta)}}
\newcommand{\V}{\mathbb{V}_{p(x;\theta)}}
\newcommand{\K}{\mathcal{K}}
\newcommand{\R}{\mathbb{R}}
\newcommand{\xx}{\frac{\langle x, x' \rangle}{\norm{x}\norm{x'}}}

\newcommand{\dellogp}{\nabla_{\theta}\log p(x;\theta)}
\newcommand{\delp}{\nabla_{\theta}p(x;\theta)}
\newcommand{\p}{p(x;\theta)}
\newcommand{\g}{g(\hat{\epsilon}, \theta)}
\newcommand{\eps}{\epsilon}
\newcommand{\nt}{\nabla_{\theta}}
\newcommand{\kernel}{k(x_i, x_j)}
\newcommand{\ft}{\mathcal{FT}}
\newcommand{\Et}{\mathbb{E}_{\theta}}
\newcommand{\one}{\mathbbm{1}}
\newcommand{\dfz}{\frac{\partial f}{\partial z}}
\newcommand{\det}{\text{det}}
\newcommand{\Cov}{\mathrm{Cov}}



\hypersetup{%
  colorlinks=true,
  linkcolor=blue,
  linkbordercolor={0 0 1}
}
 
\renewcommand\lstlistingname{Algorithm}
\renewcommand\lstlistlistingname{Algorithms}
\def\lstlistingautorefname{Alg.}

\lstdefinestyle{Python}{
    language        = Python,
    frame           = lines, 
    basicstyle      = \footnotesize,
    keywordstyle    = \color{blue},
    stringstyle     = \color{green},
    commentstyle    = \color{red}\ttfamily
}

\setlength{\parindent}{0.0in}
\setlength{\parskip}{0.05in}
%%%%%%%%%%




% if you need to pass options to natbib, use, e.g.:
%     \PassOptionsToPackage{numbers, compress}{natbib}
% before loading neurips_2019

% ready for submission
% \usepackage{neurips_2019}

% to compile a preprint version, e.g., for submission to arXiv, add add the
% [preprint] option:
%     \usepackage[preprint]{neurips_2019}

% to compile a camera-ready version, add the [final] option, e.g.:
\usepackage[final]{neurips_2019}

% to avoid loading the natbib package, add option nonatbib:
%     \usepackage[nonatbib]{neurips_2019}

\usepackage[utf8]{inputenc} % allow utf-8 input
\usepackage[T1]{fontenc}    % use 8-bit T1 fonts
\usepackage{hyperref}       % hyperlinks
\usepackage{url}            % simple URL typesetting
\usepackage{booktabs}       % professional-quality tables
\usepackage{amsfonts}       % blackboard math symbols
\usepackage{nicefrac}       % compact symbols for 1/2, etc.
\usepackage{microtype}      % microtypography

\newcommand {\john}[1]{{\color{blue}\sf{[john: #1]}}}



\newcommand{\matr}[1]{\mathbf{#1}}

\title{Correlation of the Jacobian and Probability Flow}

% The \author macro works with any number of authors. There are two commands
% used to separate the names and addresses of multiple authors: \And and \AND.
%
% Using \And between authors leaves it to LaTeX to determine where to break the
% lines. Using \AND forces a line break at that point. So, if LaTeX puts 3 of 4
% authors names on the first line, and the last on the second line, try using
% \AND instead of \And before the third author name.

\author{%
  John Y.~Shin \\
  Department of Computer Science\\
  New York University\\
  New York, NY 10033 \\
  \texttt{jys308@nyu.edu} \\
  % examples of more authors
  % Affiliation \\
  % Address \\
  % \texttt{email} \\
  % \AND
  % Coauthor \\
  % Affiliation \\
  % Address \\
  % \texttt{email} \\
  % \And
  % Coauthor \\
  % Affiliation \\
  % Address \\
  % \texttt{email} \\
  % \And
  % Coauthor \\
  % Affiliation \\
  % Address \\
  % \texttt{email} \\
}

\begin{document}

\maketitle

\begin{abstract}
\end{abstract}

\section{Computing the Jacobian}

Suppose we have a Neural Network classifier $f : \R^{d} \rightarrow \R^{k}$, where $d$ is the dimensionality of the input and $k$ is the number of classes. We define the Jacobian matrix of the Neural Network as:

\begin{equation}
J(\matr{x}) = \nabla_\matr{x} f(\matr{x}) \in \R^{k} \times \R^{d}
\end{equation}

Or the derivative of the Neural Network with respect to the input. Suppose we have a training dataset of size $n$, with examples given as $(\matr{x}_1, \matr{x}_2, \cdots, \matr{x}_n)$, where $\matr{x}_i \in \R^d$. We can concatenate the $J(\matr{x})$ for each example into a vector:

\begin{equation}
\mathcal{J}_i = \text{flatten}(J(\matr{x}_i)) \in \R^{k \times d}}
\end{equation}

We can then form a matrix of these $\mathcal{J}_i$'s into a matrix $\mathcal{J} \in \R^n \times \R^{k \times d}$, which we will call the Jacobian matrix of the training dataset. Each row corresponds to a training example and the columns are a concatenation of the derivatives of each dimension of the output with each dimension of the input.

As a classifier trained with a cross-entropy loss and with a softmax as the final output of the network, we can interpret the output of the network as a vector of probabilities:

\begin{equation}
f(\matr{x})  = (p(y_1| \matr{x}), p(y_2| \matr{x}), \cdots, p(y_k| \matr{x})) \in \R^k
\end{equation}

The Jacobian matrix of the neural network then becomes:

\begin{equation}
J(\matr{x}) = \nabla_{\matr{x}} f(\matr{x}) = (\nabla_{\matr{x}} p(y_1| \matr{x}), \nabla_{\matr{x}} p(y_2| x), \cdots, \nabla_{\matr{x}} p(y_k| \matr{x})) \in \R^k \times \R^d 
\end{equation}

For conciseness (and abuse) of notation, we can write the the flattened $\mathcal{J}_i$ in vector notation as:

\begin{equation}
\mathcal{J}_i = \nabla_{\matr{x}_i} p(\matr{y} | \matr{x}_i) \in \R^{k \times d}
\end{equation}

We can then build the correlation matrix of the Jacobian over the training dataset as:

\begin{equation}
M_{ij} = (\mathcal{J} \mathcal{J}^T)_{ij} = \nabla_{\matr{x}_i} p(\matr{y}|\matr{x}_i) \cdot \nabla_{\matr{x}_j} p(\matr{y}|\matr{x}_j) \in \R^{n \times n} 
\end{equation}

\begin{theorem}
testing
\end{theorem}

Suppose we have a random matrix, $M$, where each matrix element $M_{ij}$ is the covariance of two random variables, $\Cov(X_i, X_j)$. 


\bibliographystyle{unsrt}
\bibliography{sample.bib}



\end{document}
